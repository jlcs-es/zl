\documentclass{report}

\usepackage[utf8]{inputenc}
\usepackage{hyperref}
\usepackage{fancyvrb}
\usepackage[dvipsnames]{xcolor}

\hypersetup{
	colorlinks,
	citecolor=black,
	filecolor=black,
	linkcolor=black,
	urlcolor=black
}

% redefine \VerbatimInput
\RecustomVerbatimCommand{\VerbatimInput}{VerbatimInput}%
{
	fontsize=\footnotesize,
	%
	frame=lines,  % top and bottom rule only
	framesep=2em, % separation between frame and text
	rulecolor=\color{Gray},
	%
	% label=\fbox{\color{Black}data.txt},
	% labelposition=topline,
	%
}

\author{Ezequiel Santamaría Navarro}
\title{Trabajo fin de grado sobre la construcción de un entorno de aprendizaje para la programación.}


\renewcommand{\abstractname}{Abstracto}
\renewcommand{\contentsname}{Índice de contenido}
\renewcommand{\chaptername}{Parte}

\begin{document}
\maketitle
\tableofcontents
	
\begin{abstract}
	Abstracto
\end{abstract}


\chapter{Introducción}
Introducción

\chapter{Lenguaje zl}
\section{Introducción}
El lenguaje está enfocado a cumplir las siguientes propiedades:

\begin{itemize}
	\item Enfocado a parecerse sintáticamente al pseudocódigo en castellano, parecido al enseñado en la asignatura IP.
	\item Diferenciando los tipos de datos, sin hacer implícitas sus conversiones ni sus operaciones.
	\item Insensible a mayúsculas y minúsculas, 
	y aceptando tildes y eñes en los nombres.
	\item Con un único tipo de datos para representar números enteros y decimales (similar a Javascript).
\end{itemize}

\section{Definición formal del lenguaje}
La gramática del lenguaje en formato BNF es:

\VerbatimInput{../sintaxis.txt}

Cabe destacar de la gramática las 4 expresiones y los cuatro grupos de operadores, para poder definir correctamente el orden en el cuál los operadores se reducen.

Por otro lado, los número en el lenguaje permiten indicar base 2, 10 o 16 de la siguiente manera:

\begin{itemize}
	\item $51966$ como número en base 10
	\item $51966|10$ como el mismo número
	\item $CAFE|16$ o $cafe|16$ como el mismo número en base hexadecimal.
	\item $1100101011111110|2$ como el número en base binaria.
\end{itemize}

\end{document}

